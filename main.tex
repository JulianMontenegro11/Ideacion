\documentclass{article}
\usepackage[utf8]{inputenc}
\usepackage[spanish]{babel}
\usepackage{listings}
\usepackage{graphicx}
\graphicspath{ {images/} }
\usepackage{cite}

\begin{document}

\begin{titlepage}
    \begin{center}
        \vspace*{1cm}
            
        \Huge
        \textbf{Ideación}
            
        \vspace{0.5cm}
        \LARGE
        Mr. Genio
            
        \vspace{1.5cm}
            
        \textbf{Julian Montenegro Pinzon}\\
        
        
        \textbf{Cristian Martinez De La Ossa}
        
            
        \vfill
            
        \vspace{0.8cm}
            
        \Large
        Despartamento de Ingeniería Electrónica y Telecomunicaciones\\
        Universidad de Antioquia\\
        Medellín\\
        Marzo de 2021
            
    \end{center}
\end{titlepage}

\tableofcontents
\newpage
\section{Introducción}\label{intro}
Vivimos en tiempos donde la mayoría de niños y jóvenes dedican mucho tiempo a los videojuegos, tiempo que podrían usar de manera más productiva en su educación, podemos notar que existe una competencia entre la buena educación y el simple entretenimiento. Así que como proyecto de programación queremos tratar de equilibrar la relación entre estos dos aspectos.

Al pensar en la construcción del proyecto final nos dimos cuenta de que con ayuda de un juego podríamos apoyar el proceso de aprendizaje en diferentes campos del conocimiento como matemáticas, biología, química, ingles, etc. De esta manera el niño no tendrá un tiempo improductivo aunque este jugando.

\section{Descripción del proyecto} \label{contenido}
Una vez clara la idea principal del proyecto quicimos construir un juego basado en respuestas rápidas,  diferentes acertijos sobre diferentes áreas y también de cultura general.

Para iniciar queremos que el usuario del juego se sienta cómodo por lo que un letrero de bienvenida muy amigable es la mejor opción, allí mismo el usuario tendrá la oportunidad de dar iniciar  su experiencia  desde el nivel "novato" que tendrá un considerable banco de preguntas y el cual solo podrá ser superado cuando el usuario  pueda acertar a 15 preguntas de manera consecutiva, en cada pregunta el usuario tendrá un tiempo establecido para contestar y al elegir su opción de respuesta, podrá darse cuenta de su acierto o error con un sonido e imagen adecuada para cada caso.  Serán 3 niveles en total (novato, aprendiz y experto) a medida que el usuario avanza entre niveles se dará cuenta de que el tiempo establecido para las respuestas ira disminuyendo y la dificultad de las preguntas ira aumentando.


\newpage
\section{Conclusión}
Con nuestro proyecto estamos buscando que el obtener conocimientos pase de ser algo aburrido a ser algo divertido sin dejar de ser un desafío, con cada pregunta se tendrá una nueva oportunidad de aprender ya sea de manera individual o confrontando nuestros conocimientos con algún amigo.

\end{document}


